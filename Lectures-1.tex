\documentclass[12pt]{article}

\usepackage[utf8]{inputenc}
\usepackage[russian]{babel}

\usepackage{amssymb}
\usepackage{xcolor}

\usepackage{indentfirst}

\usepackage[normalem]{ulem}

\newcommand{\example}{{\itshape Пример. }}
\newcommand{\equals}{\Leftrightarrow}
\newcommand{\defi}{{\itshape Определение. }}
\newcommand{\exc}{{\bfseries Упражнение. }}
\newcommand{\thr}{{\bfseries Теорема. }}
\newcommand{\lem}{{\bfseries Лемма. }}

\renewcommand{\leq}{\leqslant}
\renewcommand{\geq}{\geqslant}

\begin{document}
	\title{Методы математической физики}
	\author{Сердюков Александр Сергеевич}
	\date{}
	\maketitle
	
	\section{Системы линейных дифференциальных уравнений 1 порядка}
	
	\thr Теорема существования и единственности решения задачи Коши.
	
	Если $A(t) \in C[a,b]$, то $\exists y(t) \in C^1[a,b] $ -решение задачи Коши.
	
	Схема доказательства:
	\begin{enumerate}
		\item Задачу Коши сводим к интегральному уравнению.
		\item Метод последовательных приближений.
		\item Из равномерной сходимости следует то, что мы перейти к пределу под интегралом.
	\end{enumerate}
	
	Для доказательства данной теоремы докажем пару вспомогательных лемм.
	
	\lem Пусть $y(t)$ удовлетворяет (1) и $y(t) \in C^1[a,b] \equals y(t) \in C[a,b]$ и является решением интегрального уравнения 
	$$ y(t) = y(t_0) + \int_{t_0}^t A(\tau)y(\tau)d\tau$$
	Доказательство. 
	($\Rightarrow$) $y(t) - y(t_0) = \int_{t_0}^t dy= \int_{t_0}^t A(\tau)y(\tau)d\tau$. 
	
	($\Leftarrow$) Проверяем начальные условия: $y(t_0) = y(t_0)$. По теореме о дифференцировании интеграла с пределами зависящими от параметра получаем:
	$$\frac{dy}{dt} = A(t)y(t)$$
	\raggedleft{ $\Box $}
	
	\raggedright
		
	 
\end{document}
